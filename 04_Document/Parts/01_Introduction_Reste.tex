### Rest Einleitung 
%, but not always \citep{VerHoef2007}.
% following equation \ref{eq:meanvar} \citep{Warton2012}
%\begin{equation}\tag{1.1}\label{eq:meanvar}
%	Var(Y) = \mu + \phi \mu^2 
%\end{equation} 
%Where $\mu$ is the arithmetic mean and $\phi$ a dispersion parameter.
% cite WWW 2012

%A miss-specified mean-variance relationship can lead to erroneous conclusions about species - environment interactions \citep{Warton2012}.


	%	\citet{Yee2004} argues, that, due to the exponential increase in computing power, calculating the maximum likelihood solution to RGR does not pose a problem anymore. Hence, he proposed Constrained Quadratic Ordination (CQO) which uses reduced rank vector generalized linear models (RR-VGLM) to that end.\\
%	RR-VGLM are based on reduced rank regression (termed by \citet{izenman1975reduced} but proposed by \citet{Anderson1951}), whose aim is, to reduce the $m$ original covariables to $R$ latent (i.e. unobserved) variables. $R$ is also referred to as the model's rank and is usually one or two. The reduced rank allows low-dimensional visualization and speeds up computation.
%	%Also, with ecological data, latent variables can be interpreted as environmental gradients. DIng Dong -  HUtchinson aus LL12 pdf p 10
%	VGLMs extend GLMs, in that they can handle multiple response variables and residual distributions outside the exponential family. \\
%	CQO uses a RR-VGLM whose linear predictors are restricted to be quadratic, which ensures an unimodal response as long as the regression coefficient of the quadratic term is negative.
%	Constrained Additive Ordination (CAO) \citep{Yee2006} use additive models (VGAM) instead of linear ones and is thus a more data-driven approach that is not restricted to unimodal response curves. \citet{Yee2006} advocates to use CAO for exploratory data analysis and CQO for inference, akin to using GAMs as a diagnostic tool before running GLMs \citep{Hastie2008}.\\
%	Up until now, usage of CQO and CAO is seldom in ecological studies.  
%	\citet{TerBraak2015} suggest that this is due to limitations on the number of species that can be included, a steep learning curve and numerical instability. 
%	Most applications have either been in fisheries research (e.g. \citet{Vilizzi2012}, \citet{Top2016} and \citet{Carosi2017}) or in comparisons of individual and community level species distribution models (e.g. \citet{Baselga2009} and \citet{Maguire2016}), where CQO are an instance of the latter. 

%% Why use simulations?     -- what did they do ? 
%They tested GLMs$_{mv}$s ability to compare two groups. Here, I will test the power of GLM$_{mv}$ to find true continuous gradients in simulated data and to differentiate them from noise variables.
%Simulated data have the great advantage that the "truth", i.e the relationship between environmental gradients and species abundances, is known to the scientist. The ecological model that is used to simulate such data must be constrained by current ecological theory. Simulations would be futile if the generated data had no resemblance to field data. They can however reduce the complexity inherent in the latter. If the designated methods are not capable to uncover the known truth from the simplified, simulated data set, there is little hope for them to perform better on actual ones. 
%% cite aus Austin 2006 
%The use of simulated data has been popular in studies of ordination methods \citep{}. 
%%
% Somehow insert parts of yee 2015 chapter6 intro and something akin to figure 6.1 

	%	Here, I will use simulated abundance data to test the performance of GLM$_{mv}$ to identify biologically significant continuous gradients and to differentiate them from noise variables.
%	Their performance will be compared with that of three further methods: \textit{distance based Redundancy Analysis} (db-RDA), \textit{Canonical Correspondence Analysis} (CCA) and \textit{Constrained Quadratic Ordination} (CQO).\\