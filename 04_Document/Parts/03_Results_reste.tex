%%%% ---- Reste Ergebnisee 
%% Unimodal - X 
%%GLM
%In figure \ref{fig:glm-cca-dbrda-uni} the p-values of environmental variables and noise variables are shown for  $GLM_{mv}$ (overall p-values), CCA and dbRDA when species respond unimodally to at least one the gradients. The response to the second gradient is shown with different symbols. The gap between p-values for environmental and random variables is highest in GLMs. Both environmental variables have p-values close to zero ($\mu_{env1;GLM}$: 0.002, $\mu_{env2;GLM}$: 0.003), while those of random variables have a mean of 0.63. Differences between models are not distinct.
%%CCA
%In the CCA the unimodal gradient is always ranked as statistically significant ($\mu_{env1; CCA} < 0.001$). Linear responses on the second gradient are never classified as significant. All other second response type are.  The random variable has mean of 0.196 but a median of 0.079 indicating a strong skew towards lower p-values.
%%dbRDA
%dbRDA also always recognizes env1 as statistically significant ($\mu_{env1;dbRDA} = 0.001$). Both, linear and logistic responses were not ranked as statistically significant three times. All six high p-values are from class 4 models. Despite these ($\mu_{env2;dbRDA} = 0.035$). P-values for the random variable are distributed approximately uniformly with  ($\mu_{random;dbRDA} = 0.496$).
% 
%\begin{figure}[h]
%	\centering
%	\includegraphics[width=1.1\linewidth]{../02_Figures/Mixed/GLM-CCA-dbRDA-uni}
%	\caption{P-values of environmental and noise variables for unimodal responses in $GLM_{mv}$, CCA and dbRDA. The dashed line  lies at a p-value of 0.05. The different symbols shows the response to the second gradient.}
%	\label{fig:glm-cca-dbrda-uni}
%\end{figure}
%
%%%% - Linear x 
%%GLM
%In figure \ref{fig:mixli} the p-values of GLM, CCA and dbRDA for models containing at least one linear response are displayed. The GLMs assign low p-values to both environmental variables ($\mu_{env1;GLM} = 0.002$; $\mu_{env2;GLM} = 0.002$). In models with logistic or bimodal responses to the second gradient some random variables have low p-values. All five p-values that are below 0.05 come from class three models, with many observations. The mean value of random variable p-values is 0.524, similar to that of unimodal responses.  
%% CCA
%In the presence of another response type linear responses are not classified as statistically signififcant by the CCA. Only when the responses to both gradients are linear as in Li-Li (yellow in \ref{fig:mixli}) are both recognized as statistically significant. Again, CCA has more low p-values for random variables than GLM or dbRDA.  The mean value of random variable p-values is 0.419, a lot higher than for unimodal responses.
%%dbRDA
%The performance of the dbRDA with linear gradients is similar to that of unimodal gradients. Again most p-values for environmental variables are low ($\mu_{env1;dbRDA} = 0.001$; $\mu_{env2;dbRDA} = 0.018$). The higher mean for env2 is caused by three outliers, which, are the three class four Uni-Li models that already discussed above. The mean p-value if random variables is 0.451. For p-values are below the 0.05 threshold, three of which stem from class three models and one from class two. 
%
%\begin{figure}[h]
%	\centering
%	\includegraphics[scale=1]{../02_Figures/Mixed/mix-lin}
%	\caption{P-values of environmental and noise variables for linear	 responses in $GLM_{mv}$, CCA and dbRDA. The dashed line  lies at a p-value of 0.05. The different symbols shows the response to the second gradient.}
%	\label{fig:mixli}
%\end{figure}
%
%
%%%% Logisitisch 
%%GLM 
%%CCA
%%dbRDA
%
%\begin{figure}[h]
%	\centering
%	\includegraphics[width = 1.1\linewidth]{../02_Figures/Mixed/mix_lo_box}
%	\caption{P-values of environmental and random variables for  logistic responses in $GLM_{mv}$, CCA and dbRDA. The dashed line lies at a p-value of 0.05.}
%	\label{fig:mix_lo}
%\end{figure}
%
%
%
%
%
%
%
%
%
%As species specific results are unique to $GLM_{mv}$s their resuts are shwon here without reference to other methods. 
%
%
%
%
%

		%: $Mean\ \pm\ SD\ \scriptscriptstyle env1 \textstyle = 0.002\ \pm  0.001$ and $Mean\ \pm\ SD\ \scriptscriptstyle env2 \textstyle = 0.003 \pm 0.002$.\\
				% ($Mean\ \pm\ SD\ \scriptscriptstyle env2 \textstyle = 0.579\ \pm  0.27 $).\\
				
						%$Mean\ \pm\ SD_{\scriptscriptstyle Uni/Bi-x\ env1} =  0.932 \pm 0.109 $ and 
				%$Mean\ \pm\ SD_{\scriptscriptstyle Uni/Bi-x\ env2} =  0.998 \pm 0.003 $. 
				
						%In Bi-Bi the shift occurs in \textit{env1} and \textit{env2} but the higher level in \textit{env2} is at the level of the lower one from \textit{env1}. \\ 