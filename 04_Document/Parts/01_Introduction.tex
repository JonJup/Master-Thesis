% !TeX spellcheck = en_US

%------------------------------%
%&			To Do
%------------------------------%


\section{Introduction}

	%------------------------------%
	%&			Erste Absatz:
	%&	Eternal Question + Statisticians point of view 
	%------------------------------%

	Which environmental gradients drive the changes in species abundances and community composition?
	%
	This is one of the oldest questions in ecology \citep[e.g.][]{Clements1907}  and the prospect of humans altering their surroundings at an unprecedented rate endows it with a new urgency \citep{pacifici2015assessing}.
	%
	To answer 	it, ecologists typically record the abundance or occurrence of different taxa and several environmental variables (e.g. precipitation or exposure to stressors), at different sites.
	%
	This results in a sites-by-species matrix $\mathbf{Y}$ containing multivariate species abundances, which is then statistically related to  a sites-by-predictor matrix $\mathbf{X}$, containing the environmental variables.
	%
	From a statisticians point of view, $\mathbf{Y}$ has many undesirable properties: 
	correlation within and between variables, 
	e.g. through biotic interactions \citep{morales2015inferring},
	probability distributions other than the normal, 
	more species than sites \citep[\textit{high dimensionality}, especially in DNA Barcoding studies, ][]{cristescu2014barcoding}  
	and many zeros, since most species are commonly absent from most sites (\textit{sparsity}). \\
	
	%------------------------------%
	%&		Zweite Absatz:
	%&	Distance Based + Whats wrong with it
	%------------------------------%
	
	
	Multivariate species abundances are frequently analyzed by means of their distance or dissimilarity matrix \citep[distance-based analysis \textit{sensu}][]{Warton2012}. 
	%
	This approach is popular among ecologists because it is non-parametric and hence distribution-free \citep[e.g.][]{clarke1993non}.
	% question ist schlecht 
	Whether a distance metric is appropriate depends on the properties of the data and the aim of the study, as each metric extracts different information from the raw data. 
	%
	The choice is complicated by the vast amount of available metrics \citep[see][]{Legendre2012} and contradicting recommendations \citep[][]{faith1987compositional}.
	%
	Deploying a distance metric, one also implicitly assumes a mean-variance relationship in the data. For example, the Minkowski distances (e.g. Manhattan and Euclidean) assume a constant variance across all mean values \citep{TerBraak1988}.  
	%
	However, species abundances often show a quadratic mean-variance relationship \citep{routledge1991taylor, yamamura1999transformation}.
	%
	Miss-specifying the relationship by choosing an improper distance metric can lead to erroneous conclusions about one's data, as was shown by \citet{Warton2012}.
	%
	An alternative to distance-based analysis, that avoids this issue, is the model-based approach.\\ 
 
	%------------------------------%
	%&		Dritte Absatz:
	%&	Model Based Approach
	%------------------------------%


	The model-based approach to multivariate data analysis entails explicitly specifying a statistical model of the process that generated the observed data \citep{Warton2015a}.
	%
	This includes the mean-variance relationship, which can be adjusted to the properties of the data.
	%
	Despite their ubiquity in univariate analyses, model-based approaches have long been uncommon in multivariate ecological analyses.
	%
    However, advances in statistical theory and computation power have led to a surge of models for multivariate abundance data. 
	%
	Recent examples include \textit{Hierarchical Modeling of Species Communities} \citep[HMSC,][]{Ovaskainen2017}, \textit{Generalized Joint 
	%
	Attribute Modelling} \citep[GJAM,][]{Clark2017} and \textit{multivariate Generalized Linear Models} \citep[GLM$_{mv}$,][]{Warton2012}.\\
	
	%------------------------------%
	%&		Vierte Absatz:
	%&			GLMmv
	%------------------------------%
	
	In GLM$_{mv}$, a separate univariate GLM is fit to each taxon, each model using the same predictors. 
	%
	Univariate GLMs are a powerful and flexible method. 
	%
	They are strongly advocated for the analysis of count or occurence data as they can handle different residual distributions and mean-variance relationships \citep{OHara2010, Warton2011, Szocs2015_b}.
	%
	Extending them to multi-species abundance data was thus a natural starting point for multivariate model-based analyses \citep{Warton2012}.
	%
	The univariate models are combined by  summing their test statistics, which enables the researcher to draw conclusions about the whole community.
	%
	GLM$_{mv}$ were one of the earlier multivariate models with an easy-to-use implementation \citep[in the \textit{mvabund} R-package, ][]{Wang2018} and  
	since their introduction they have gained traction within the ecological community (244 citations according to the Web of Science as of 24.04.2018). 
	%
	To my knowledge, the simulations of \citet{Warton2012} remain the only test of GLM$_{mv}$ with simulated data until now.
	%
	\citet{Szocs2015} tested GLM$_{mv}$ with data from ecotoxicological mesocosm studies, and found that they performed better or at least as well as commonly used methods (Principal Response Curves).
	%
	They also emphasized the need for further simulation studies on GLM$_{mv}$.\\

	%------------------------------%
	%&		5. Absatz:
	%&		Aim of the Study
	%------------------------------%
	
	The aim of this study is to test the ability of four statistical methods to determine which environmental gradients drive the changes in sets of simulated multivariate abundance data. 
	%
	The performance of GLM$_{mv}$ will be compared to \textit{Distance Based Redundancy Analysis} (db-RDA), \textit{Canonical Correspondence Analysis} (CCA) as well as \textit{Constrained Additive} and \textit{Quadratic Ordination} (CAO/ CQO).

	%------------------------------%
	%&		6. Absatz:
	%&		Other Methods
	%------------------------------%
	
	% db-RDA
	db-RDA is a distance-based analysis. 
	%
	It calculates an ordination on the distance matrix of the sites-by-species  matrix $\mathbf{Y}$ constrained by the environmental variables $\mathbf{X}$ \citep{Legendre1999, AndersonMati2003}.
	%
	db-RDA was highlighted by \citet{Szocs2015}, because the possibility to use asymmetrical distance metrics and avoid the \textit{double-zero problem } \citep{Legendre2012} makes them attractive for sparse data sets. 
	% CCA
	CCA and CQO are both solutions to Constrained Gaussian Ordination (CGO), in which species are expected to respond unimodally to latent gradients that are linear combinations of environmental variables \citep{gauch1974ordination}. 
	%
	CCA is an algorithmic solution (neither distance- nor model-based), that is based on the findings of \citet{TerBraak1986}. 
	%
	He showed that the maximum likelihood estimation of CGO for Poisson distributed counts can be approximated by correspondence analysis, given that a set of restrictive assumption hold (see section 2.4). 
	%
	CCA is one of the most widely used statistical techniques in ecology, with the essential papers having accumulated over 3000 citations \citep{Braak2014}.
	% CAO/ CQO
	CQO is the maximum likelihood solution to CGO \citep{Yee2004}. 
	%
	It uses an extension to GLM, \textit{Vector Generalized Linear Models} \citep[VGLM,][]{yee1996vector}, to estimate model parameters and is hence model-based. 
	%
	\citet{Yee2006} proposed CAO as a modification to CQO, that uses additive instead of linear models and is thus more flexible and data-driven. \\
	


